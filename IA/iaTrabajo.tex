\documentclass[12pt, a4paper, twoside]{article}
\usepackage[utf8]{inputenc}
\usepackage[spanish]{babel}
\usepackage{graphicx}
\usepackage{fancyhdr}
\usepackage{enumitem}

\pagestyle{fancy}
\fancyhf{}
\fancyfoot[R]{\thepage}
\fancyfoot[L]{\nouppercase{\leftmark}}
\fancyhead[R]{Ia y Empleo.}
\fancyhead[L]{Carlos G. Pérez Aranda. }

\graphicspath{{./images/}}
\title{IA y Empleo}
\author{Carlos G. Pérez Aranda}

\begin{document}
\maketitle
\newpage
\section{Estudio de las Fuentes}
Según la información de las fuentes, la relación entre la creación y eliminación de puestos de trabajo debido a la Inteligencia 
Artificial (IA) es un tema complejo y con diferentes perspectivas y proyecciones.
Por un lado, algunas fuentes sugieren que si bien la automatización y los robots reemplazarán muchos trabajos existentes, 
al mismo tiempo se crearán multitud de nuevos puestos\cite{rouhiainen2018inteligencia}. Se menciona que hay un consenso generalizado en que la IA creará más empleos 
de los que reemplazará\cite{tenes2023impacto}. Datos más recientes de una encuesta del World Economic Forum en 2023 indican que se espera que la adopción de 
de la IA por parte de las empresas (cerca del 75\% de ellas) cause una alta rotación de personal, generando nuevas oportunidades de 
trabajo para el 50\% de las organizaciones, aunque también pérdidas de empleo para un 25\%\cite{tenes2023impacto}.
Sin embargo, otras fuentes presentan un panorama más centrado en el desplazamiento y la transformación del trabajo:

Se afirma que la automatización gracias a la IA puede hacer que algunas funciones de trabajo sean innecesarias, cambiando la forma 
en que los empleados realizan sus tareas\cite{tenes2023impacto}\cite{valverde2019aplicaciones}.

Un análisis citado por una fuente estima que la IA dejará sin trabajo a 300 millones de personas. Aquellos que conserven sus 
trabajos podrían ver su carga de trabajo reducida a la mitad\cite{tenes2023impacto}.

Se estima que hasta un tercio de las actividades laborales actuales podrían verse afectadas por la automatización para 2030\cite{tenes2023impacto}.

Cerca de la mitad de las actividades laborales a nivel global podrían ser teóricamente automatizadas con las tecnologías disponibles 
actualmente. Menos del 5\% de las ocupaciones consisten en actividades que podrían ser completamente automatizadas, pero en alrededor del 
60\% de las ocupaciones, al menos un tercio de las actividades podrían automatizarse, lo que implicaría transformaciones sustanciales\cite{tenes2023impacto}.

Se prevé que la IA y la automatización tienen el potencial de impactar en prácticamente todas las industrias\cite{tenes2023impacto}.

Se habla de un cambio considerable en el panorama laboral, causando trastornos en carreras consolidadas y reduciendo trabajadores 
humanos en ciertos sectores, lo que se describe como un ``desplazamiento laboral masivo''\cite{tenes2023impacto}.

Este desplazamiento afecta especialmente a trabajos repetitivos y basados en reglas, y se señala su potencial impacto negativo en 
Asia y África, donde los trabajos tradicionales de baja cualificación podrían ser reemplazados por máquinas inteligentes\cite{tenes2023impacto}.

La IA ha reconfigurado la mano de obra humana, ``a menudo en detrimento de los trabajadores''\cite{tenes2023impacto}.
Aunque haya proyecciones que indican una creación neta de empleos en general o dentro de las empresas que adoptan la IA\cite{rouhiainen2018inteligencia}\cite{tenes2023impacto}, las fuentes 
enfatizan que habrá una transformación significativa de los empleos existentes\cite{martinez2020tecnologias}, con la automatización de actividades, y un desplazamiento 
laboral masivo que afectará a muchos trabajadores, especialmente aquellos con tareas automatizables. El futuro laboral se inclinará más 
hacia roles creativos y cognitivos que complementen a la IA, y requerirá nuevas habilidades, tanto técnicas como humanas\cite{tenes2023impacto}.
En resumen, mientras algunas fuentes citan un consenso o proyecciones que sugieren que la IA creará más empleos de los que eliminará, 
otras ponen un fuerte énfasis en el potencial de desplazamiento significativo, la reducción de trabajadores en ciertos roles y la profunda 
transformación de la naturaleza del trabajo para la mayoría de las ocupaciones. La clave para navegar este cambio radica en la adaptación y 
la adquisición de las habilidades requeridas.

\section{Opinión personal}

Aunque es innegable que la IA provocará una transformación profunda del mercado laboral, considero que creará más empleos de los que eliminará, por las siguientes razones:

\begin{enumerate}
\item La historia de la tecnología respalda esta tendencia
A lo largo de la historia, cada revolución tecnológica —desde la máquina de vapor hasta Internet— ha generado temores similares sobre la pérdida masiva de empleos. Sin embargo, en todos los casos, la tecnología terminó creando más trabajos de los que destruyó, aunque diferentes en naturaleza. La IA no es una excepción, sino una continuación de este patrón.

\item La IA no elimina trabajos, transforma tareas
Según las fuentes, menos del 5% de los empleos pueden ser completamente automatizados. En cambio, en el 60% de los trabajos, solo una parte de las tareas puede ser automatizada. Esto significa que la IA libera tiempo para que los trabajadores se enfoquen en tareas más creativas, estratégicas o humanas, en lugar de reemplazarlos por completo.

\item Nuevas industrias y roles emergentes
La adopción de la IA está dando lugar a nuevas industrias (como la ética de IA, el entrenamiento de modelos, la ingeniería de prompts, la supervisión de algoritmos, etc.) y a nuevos roles que antes no existían. Estas oportunidades laborales requieren habilidades distintas, pero son cada vez más demandadas.

\item La IA como herramienta de productividad
La IA puede aumentar la productividad de los trabajadores, permitiendo que las empresas crezcan más rápido y generen más empleo. Por ejemplo, un diseñador asistido por IA puede producir más contenido en menos tiempo, lo que puede traducirse en mayores ingresos y expansión de equipos.

\item Datos recientes apoyan esta visión
Según el World Economic Forum (2023), el 75% de las empresas planea adoptar IA, y el 50% espera que esto genere nuevas oportunidades laborales. Aunque un 25% prevé pérdidas de empleo, el saldo neto sigue siendo positivo. Además, hay un consenso creciente en que la IA creará más empleos de los que reemplazará, especialmente en sectores que la integren de forma estratégica.

\item La clave está en la adaptación
El verdadero desafío no es la destrucción de empleos, sino la reconversión laboral. Con políticas adecuadas de formación y reciclaje profesional, los trabajadores podrán adquirir las habilidades necesarias para ocupar los nuevos puestos que la IA generará. Esto convierte a la IA en una oportunidad de evolución profesional, no en una amenaza.

\end{enumerate}

\section{Conclusión}

La IA no es un fin del trabajo humano, sino una transformación de su naturaleza. Si bien el desplazamiento laboral es real y debe ser gestionado con responsabilidad, la capacidad de la IA para generar nuevas industrias, aumentar la productividad y liberar el potencial humano es aún mayor. Por eso, creo firmemente que la IA creará más empleos de los que destruirá, siempre que sepamos adaptarnos y preparar a la sociedad para este cambio.


\bibliographystyle{plain} % Choose a bibliography style (e.g., plain, alpha, etc.)
\bibliography{bibliografia} % Replace 'bibliography' with the name of your .bib file (without the extension)

\end{document}